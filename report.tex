\documentclass{beamer}
\usepackage{styles/fonts}
\usepackage{svg}

\begin{document}
\title{Задание по курсу ``Введение в ркт''}   
\author{Королев А. К. 8О-402Б} 
\date{\today} 

\frame{\titlepage} 

% \frame{\frametitle{Table of contents}\tableofcontents} 


\section{Section no.1} 
\frame{\frametitle{F-35} 
  \small
Семейство малозаметных многофункциональных истребителей-бомбардировщиков пятого поколения, разработанное американской фирмой Lockheed Martin.
\begin{figure}
\includegraphics[width=0.9\textwidth]{f35cool.jpg}
\end{figure}
}
\subsection{Subsection no.1.1  }


\frame{\frametitle{Характеристики}
  \normalsize
  % \def\arraystretch{1.5}
  \begin{tabular}[c]{|l|c|}
    \hline
    Длина
     & $15.57 \text{ м}$ \\
    \hline
    Ширина размаха крыльев
     & $10.67 \text{ м}$ \\
    \hline
    Высота
     & $4.38 \text{ м}$ \\
    \hline
    Площадь крыла
     & $42.7 \text{ м}^{2}$ \\
    \hline
    Максимальная скорость
     & $1930 \text{ км/ч}$ \\
    \hline
    Крейсерская скорость
     & $850 \text{ км/ч}$ \\
    \hline
    Дальность полета
     & $2200 \text{ км}$ \\
    \hline
    Максимальная взлетная масса
     & $29100 \text{ кг}$ \\
    \hline
  \end{tabular}


  \footnotesize

  \vspace{0.5cm}
    Дальность полета 2200 $\text{ км}$ соответствует ближнемагистралным самолетам (в терминологии гражданской авиации).

    Взлетная масса $29100 \text{ кг}$ соответствует 3 классу по взлетной массе.
}

\subsection{Lists II}
\frame{\frametitle{Число маха}
  \begin{align*}
    M = \frac{v}{a} =
    &\frac{1930 \text{ км/ч}}{1191 \text{ км/ч}} = 1.62 \in  [1.2, 5] \implies \\
    &\implies \textit{Сверхзвуковой}
  \end{align*}
}
\frame{\frametitle{Форма крыла}
\begin{columns}
\column{0.5\textwidth}
\small
  Форма крыла - трапециевидная.

  Стреловидность по передней кромке:
  $ \chi _v = \arctan \frac{84}{125} = 0.6 \approx 33 ^{\circ} $

  Стреловидность по задней кромке:
  $ \chi _h = - \arctan \frac{35}{125} = - 0.27 \approx - 16 ^{\circ} $

  Поперечный угол $\psi \approx 0$:
  \scriptsize
\begin{figure}
\includesvg[width=\textwidth]{f35front.svg}
\end{figure}
\column{0.5\textwidth}
\begin{figure}
\includesvg[width=\textwidth]{main.svg}
\end{figure}
\end{columns}
% \begin{center}
% \includegraphics[height=3.0in]{f35single.png}
% \end{center}
}

\frame{\frametitle{Классификация}
  \normalsize
  % \def\arraystretch{1.5}
  \begin{tabular}[c]{|l|c|}
    \hline
    по числу крыльев & моноплан \\
    \hline
    по расположению крыльев & среднеплан \\
    \hline
    по типу и расположению оперения & хвостовое оперение \\
    \hline
  \end{tabular}
}

\subsection{Tables}
\frame{\frametitle{Сужение крыла}
\begin{columns}
  \normalsize
  $\eta = \frac{b_{\text{кр}}}{b_{\text{кн}}} = \frac{300}{70} = 4.28  $
\column{0.5\textwidth}
\begin{figure}
\includesvg[width=\textwidth]{f35wing.svg}
\end{figure}
\end{columns}
}

\frame{\frametitle{Удлинение крыла}
\begin{columns}
  \normalsize
  $\lambda = \frac{l}{b_{\text{ср}}} = \frac{490}{200} = 2.45$
\column{0.5\textwidth}
  \footnotesize
\begin{figure}
\includesvg[width=\textwidth]{f35wing2.svg}
\end{figure}
  \normalsize
\end{columns}
}

\frame{\frametitle{Потребная тяга}
  \normalsize
  $G_{\text{ср}} \approx 25000 \text{ кг}$

  $K \approx 10$

  $P_{n} = \frac{G_{\text{ср}}}{K} = \frac{25000}{10} = 2500 \text{ кг}  $

}

\end{document}

%%% Local Variables:
%%% mode: latex
%%% TeX-master: t
%%% TeX-engine: luatex 
%%% TeX-command-extra-options: "-auxdir=/tmp"
%%% End: