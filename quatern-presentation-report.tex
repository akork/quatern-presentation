\documentclass{beamer}

\usepackage{styles/fonts}
% \usepackage{styles/sections}
% \usepackage{styles/tcb}
% \usepackage{styles/colors}
% \usepackage{styles/math}
% \usepackage{styles/misc}

\begin{document}

\title{Выпускная квалификационная работа 
на тему:

Интегрирование кинематических уравнений движения твердого тела в кватернионах.}   
\author{Королев А. К. 8О-402Б} 
\date{\today} 

\frame{\titlepage} 

% \frame{\frametitle{Table of contents}\tableofcontents} 


\section{Section no.1} 
\frame{\frametitle{Введение}
  Возможные способа задания ориентации твердого тела:
  \begin{itemize}
  \item углы эйлера
  \item матрица поворота (направляющих косинусов)
  \item кватернион
  \end{itemize}
  
}
\begin{frame}{Motivation}
  Углы эйлера
  Ориентацию тела можно задавать тремя углами поворота вокруг осей координат.
  Последовательные повороты должны производиться вокруг разных осей. Всего таких
  комбинаций поворотов \( 3 \cdot 2 \cdot 2 = 12 \). Например, комбинация ZXZ
  соответствует классическим углам Эйлера (прецессии, нутации, собственного
  вращения);
  комбинация ZXY - самолетным углам (рысканье, тангаж, крен).
  \begin{figure}
    \includegraphics[width=0.8\textwidth]{A4angles.png}
    % \includegraphics[width=0.4\textwidth]{angles-ZXY.png}
  \end{figure}
\end{frame}
\begin{frame}{Минусы углов Эйлера}
  Существует проблема Gimbal Lock.
  Существуют особые положения в которых число степеней свободы уменьшается на 1,
  то есть из этих положений движение по одному из направлений невозможно.
  Например, при угле нутации \( \theta = 0, \pi \) изменение других углов
  приводит к вращению относительно одной и той же оси Z. 
  \begin{figure}
    \includegraphics[width=0.8\textwidth]{gimble.png}
    % \includegraphics[width=0.4\textwidth]{angles-ZXY.png}
  \end{figure}
  
\end{frame}

\end{document}

%%% Local Variables:
%%% mode: latex
%%% TeX-master: t
%%% TeX-engine: luatex 
%%% End:
